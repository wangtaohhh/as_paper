\documentclass[a4paper]{article}
\title{马尔科夫状态模型协助发现靶点的隐藏口袋}                   %———总标题
   \author{王韬 2024221120}

\usepackage{ctex}
\usepackage{CJKutf8}
\renewcommand{\abstractname}{\textbf{\zihao{4}摘要}}
\begin{document}
	
\begin{CJK}{UTF8}{gbsn}
\CJKfamily{song}
\maketitle

\begin{center}
\tableofcontents
\end{center}

 \begin{abstract}

基于结构的药物设计(SBDD)需要预先了解特定靶点的药物结合位置,目前现有的计算方法大都通过溶液可及表面积识别蛋白空腔,而现实中的蛋白质不断运动,在运动的过程中很可能产生结构解析中无法发现的隐藏口袋。遗憾的是,大多数隐藏的口袋位点只有在解析到一个小分子可以稳定结合时才会被发现。通过课程的学习,我提出了一种方法,不使用结构生物学的方法发现隐藏的药物结合位点口袋,而是利用马尔可夫状态模型,通过寻找马尔科夫状态的波动间接发现隐藏口袋。这种方法弥补了高通量筛选或者结构生物学“高成本,长周期”的缺点,为药物设计提供一种计算手段。

 \end{abstract}
\newpage


\section{马尔科夫状态模型的构建过程}
	\subsection{K-Medoids聚类}

K-Medoids算法在K-means算法的基础上衍生而来,K-Means聚类算法的目的是将分子动力学(MD)轨迹数据集划分为K个不重叠的聚类$\{C_{1},C_{2}, \cdots,  C_{k}\}$,使不同的状态与相应聚类中心(几何平均值)的距离平方和最小化。其算法可以表示为

\begin{equation}
min\sum_{i=1}^{K}\sum_{x\in C_{i}}^{} {\vert\vert x- \mu_{i} \vert\vert}^{2}
\end{equation}

其中,$x$为第$i$个聚类$C_{i}$中的MD构象,$\mu_{i}$为聚类中心。


这种方法可以对不同的MD数据进行聚类,生成“微状态”代表一部分轨迹。但这种方法的缺点在于生成的微状态并不一定是动力学过程中的一帧轨迹,原因在于K-Means方法对轨迹计算了平均值。在此,我使用K-Medoids算法进行聚类,这种方法与K-Means的区别在于K-Medoids的聚类中心选取了模拟的一帧轨迹,而非轨迹的均值,其算法可以表示为

\begin{equation}
min\sum_{i=1}^{K}\sum_{x\in C_{i}}^{} {\vert\vert x- \mathrm C(C_{i}) \vert\vert}^{2}
\end{equation}


其中,$\mathrm C(C_{i})$是作为聚类中心的构象,而非均值。

通过以上算法,我们可以将MD轨迹数据划分为一个一个的“微状态”,供后续马尔科夫模型构建使用。

	\subsection{计算转移矩阵}
	\subsection{构建马尔科夫状态模型}
	\subsection{查询局部波动}

\section{发现蛋白的隐藏位点}
	\subsection{分子动力学模拟}
		\subsubsection{蛋白选择}
		\subsubsection{模拟细节}
	\subsection{隐藏位点的寻找}




\end{CJK}
\end{document}